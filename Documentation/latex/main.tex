\documentclass[openright,twoside,10pt]{book}
\usepackage[b5paper,left=2cm,top=2.5cm,right=1.5cm,bottom=2.5cm]{geometry} 
\usepackage[spanish, es-tabla]{babel} % espanol
\usepackage[utf8]{inputenc} % acentos sin codigo
\usepackage{graphicx} % gráficos
\usepackage{lscape}
\usepackage{fancyvrb}
\usepackage{fancyhdr}
\usepackage{wrapfig}
\usepackage[hidelinks]{hyperref}
\usepackage{biblatex}
\bibliography{bibliografia}
\usepackage{float}
\usepackage{libertine}
\usepackage{csquotes}

\providecommand{\tightlist}{%
  \setlength{\itemsep}{0pt}\setlength{\parskip}{0pt}}
  
\setlength{\parskip}{10pt plus 1pt minus 1pt}
 % aqui definimos el encabezado de las paginas pares e impares.
\rhead[]{}

\renewcommand{\headrulewidth}{0.5pt}

% aqui definimos el pie de pagina de las paginas pares e impares.
\rfoot[\thepage]{\thepage}
\cfoot[]{}
\renewcommand{\footrulewidth}{0pt}

%redefino el verbatim
%\renewenvironment{verbatim}{\begin{Verbatim}[frame=single,fontsize=\small]}{\end{Verbatim}}


% aqui definimos el encabezado y pie de pagina de la pagina inicial de un capitulo.
\fancypagestyle{plain}{
\fancyhead[R]{}
\fancyfoot[C]{}
\fancyfoot[R]{\thepage}
\renewcommand{\headrulewidth}{0.5pt}
\renewcommand{\footrulewidth}{0pt}
}

\pagestyle{fancy} % seleccionamos un estilo

\date{$fecha$}
\author{$autor$}
\title{$titulo$}

\begin{document}
    {
        \fontfamily{phv}\selectfont
        \begin{titlepage}
        \begin{center}
            \vspace*{-1in}
            \begin{figure}[htb]
                \begin{center}
                    \includegraphics[width=3cm]{./latex/img/logo}
                \end{center}
            \end{figure}
            \begin{Large}
                \textbf{Universidad de Valladolid}
            \end{Large}

            \vspace*{0.15in}
            \vspace*{0.6in}
            \begin{Huge}
                {Escuela de Ingeniería Informática\\}
            \end{Huge}
            \vspace*{0.2in}
            \begin{Large}
                \textbf{\textsc{Trabajo Fin de Grado\\}}
            \end{Large}
            \vspace*{0.5in}
            \begin{Large}
                { Grado en Ingeniería Informática}\\
                { Mención en $mencion$ \\}
            \end{Large}
            \vspace*{0.5in}
            %\rule{140mm}{0.1mm}\\
            \vspace*{0.3in}
            \begin{large}
                \textbf{{\LARGE $titulo$\\}}
            \end{large}
            \vspace*{0.3in}
            %\rule{140mm}{0.1mm}\\
            \vspace*{1.3in}
            \begin{large}
                \begin{flushright}
                    Autor:\\
                    \textbf{$autor$} \\
                    \vspace*{0.3in}
                    Tutor:\\
                    \textbf{$tutor$}
                \end{flushright}
            \end{large}
        \end{center}
    \end{titlepage}

    }
    
    \newpage
    \mbox{}	
    \thispagestyle{empty} % para que no se numere esta página

    \chapter*{}
    \pagenumbering{Roman} % para comenzar la numeración de paginas en números romanos

    \begin{flushright}
        \textit{%Dedicatoria,\\
        $dedicatoria$}
    \end{flushright}

    \chapter*{Agradecimientos} % si no queremos que añada la palabra "Capitulo"
    \addcontentsline{toc}{chapter}{Agradecimientos} % si queremos que aparezca en el índice
    \markboth{AGRADECIMIENTOS}{AGRADECIMIENTOS} % encabezado 

    $agradecimientos$

    \chapter*{Resumen} % si no queremos que añada la palabra "Capitulo"
    \addcontentsline{toc}{chapter}{Resumen} % si queremos que aparezca en el índice
    \markboth{RESUMEN}{RESUMEN} % encabezado
    %\begin{flushleft}

    $resumen$

    %\end{flushleft}


    \chapter*{Abstract} % si no queremos que añada la palabra "Capitulo"
    \addcontentsline{toc}{chapter}{Abstract} % si queremos que aparezca en el índice
    \markboth{ABSTRACT}{ABSTRACT} % encabezado
    \begin{flushleft}

    $abstract$

    \end{flushleft}

    \tableofcontents % indice de contenidos

    \cleardoublepage
    \addcontentsline{toc}{chapter}{Lista de figuras} % para que aparezca en el indice de contenidos
    \listoffigures % indice de figuras

    \cleardoublepage
    \addcontentsline{toc}{chapter}{Lista de tablas} % para que aparezca en el indice de contenidos
    \listoftables % indice de tablas

    $body$

    \cleardoublepage
    %\renewcommand\bibname{Referencias Web}

    %\begin{thebibliography}{X}
    %    \bibitem{ref1} \textit{Ejemplo}, \\
    %    \textsc{ejemplo.com}.
    %    \\Recuperado a tal fecha, \\de \href{http://ejemplo.com}
    %\end{thebibliography}
\end{document}